\resheading{项目经历}
  \begin{itemize}[leftmargin=*]
    % \item
    %   \ressubsingleline{深度网络压缩与量化}{Python/算法开发}{2017.02 -- 2017.04}
    %   {\small
    %   \begin{itemize}
    %     \item 开发环境: 北京航空航天大学电子信息工程学院
    %     \item 项目属性:深度学习,算法优化
    %     \item 项目简介: 基于Caffe与Tensorflow框架复现了ShuffleNet,MobileNet,BNN等主流轻量化网络,以及Deep Compression模型压缩方法
    %     % 项目隶属于课题: 制造业产品设计服务与产业链协同技术研发与应用示范$\,$(国家支撑计划)。
    %   \end{itemize}
    %   }
    \item
      \ressubsingleline{穿戴式肌电交互设备}{项目创始人/算法开发}{2017.03 -- 2019.05}
      {\small
      \begin{itemize}
        \item 开发环境: 北京航空航天大学
        \item 项目属性:生物电信号采集处理,深度学习,嵌入式开发
        \item 项目简介: 完成了穿戴式肌肉电采集、判别设备的软硬件设计。本人负责项目企划、协调以及算法设计,基于决策树与LSTM算法设计了肌电信号判别算法。该项目参与国家"大创",并获得北航"冯如杯"科技竞赛一等奖以及创业金奖,"互联网+"大学生创新创业竞赛二等奖
      \end{itemize}
      }
    \item
    \ressubsingleline{目标识别网络SSD软硬件协同优化设计}{算法开发/FPGA开发}{2018.05 -- 2018.11}
    {\small
    \begin{itemize}
      \item 开发环境: 北京航空航天大学电子信息工程学院
      \item 项目属性:目标检测,软硬件协同设计,FPGA
      \item 项目简介: 基于PyTorch框架对目标检测网络SSD实现了剪枝、量化等网络压缩方法,并基于Verilog语言设计并实现了运算架构,进行了软硬件协同优化
    \end{itemize}
    }
    \item
    \ressubsingleline{高效神经网络架构搜索与优化方法研究}{算法开发}{2019.05 -- Present}
    {\small
    \begin{itemize}
      \item 开发环境: 清华大学电子工程系
      \item 项目属性:深度学习,神经网络架构搜索,神经网络剪枝
      \item 项目简介: 提出了可微分的高效神经网络结构化剪枝方法与基于图的神经网络架构编码性能预测器,发表于ECCV2020
    \end{itemize}
    }
	% \item
    %\ressubsingleline{双路FM语音同传系统}{嵌入式开发/FPGA开发}{2019.05 -- 2019.09}
    %{\small
    %\begin{itemize}
    %  \item 开发环境: 北京航空航天大学
    %  \item 项目属性: FM通信,数字信号处理
    %  \item 项目简介: 全国大学生电子设计竞赛项目,完成了双路FM语音通信系统,负责单片机以及部分FPGA模块设计与实现。
    %\end{itemize}
    %}

  \end{itemize}
